La ricerca operativa, dall'inglese \textit{operational research}, è una disciplina scientifica relativamente 
giovane, nata con lo scopo di fornire strumenti matematici di supporto ad attività decisionali in cui occorre gestire e coordinare 
attività e risorse limitate. La ricerca operativa permette infatti di trovare, mediante la formalizzazione 
matematica di un problema, una soluzione ottima o ammissibile (quando possibile) al problema stesso e costituisce di fatto un approccio scientifico alla risoluzione di problemi 
complessi che ha trovato grande applicazione in moltissimi ambiti, non ultimo quello industriale. 

Una delle diverse branche che compongono la ricerca operativa è l'ottimizzazione. 
Quest'ultima si occupa principalmente di problemi che comportano la minimizzazione o la massimizzazione di una funzione, detta funzione
obbiettivo, sottoposta ad un dato insieme di vincoli. 

L'oggetto del presente studio è stata una particolare classe di problemi di ottimizzazione, il \textit{vertex cover} minimo, anche detto problema di copertura dei vertici. In particolare, ne è stato analizzato l'andamento della complessità di risoluzione per alcune tipologie di grafi, generati casualmente, al variare dei parametri utilizzati nella generazione stessa.

La stesura di questo elaborato è stata articolata in quattro capitoli. Inizialmente, verranno brevemente esposti alcuni concetti teorici rilevanti nel contesto degli 
esperimenti svolti. La trattazione proseguirà poi con la presentazione dell'impostazione utilizzata nello svolgimento degli
esperimenti e dei risultati ottenuti. Infine, saranno tratte alcune conclusioni riguardo il lavoro svolto. 

Tutto il codice sviluppato in relazione a questo lavoro è stato scritto in Python \cite{python}, linguaggio di programmazione \textit{general-purpose} di alto livello che ha permesso di gestire ognuna delle diverse fasi in cui si è articolato lo svolgimento degli esperimenti, dalla generazione dei grafi all'elaborazione grafica dei risultati. Il codice sorgente, così come tutti i risultati delle elaborazioni, sono liberamente consultabili online nella \textit{repository} utilizzata per gestire il controllo di versione del progetto \cite{repository}.