La ricerca operativa, dall'inglese \textit{operational research}, è una disciplina scientifica relativamente 
giovane, nata con lo scopo di fornire strumenti matematici di supporto ad attività decisionali in cui occorre gestire e coordinare 
attività e risorse limitate. La ricerca operativa permette infatti di trovare, mediante la formalizzazione 
matematica di un problema, una soluzione ottima o ammissibile (quando possibile) al problema stesso. 
Costituisce di fatto un approccio scientifico alla risoluzione di problemi 
complessi, che ha trovato grande applicazione in moltissimi ambiti, non ultimo quello industriale. 

Una delle diverse branche che compongono la ricerca operativa è l'ottimizzazione. 
Quest'ultima si occupa principalmente di problemi che comportano la minimizzazione o la massimizzazione di una funzione, detta funzione
obbiettivo, sottoposta ad un dato insieme di vincoli. 
Un problema di ottimizzazione può essere formulato come 
\begin{align}
	\label{eq:opt}
	\begin{array}{l}
      \text{min(\textit{or} max)}f(x)\\
      S	\\
      x \in D
    \end{array}
\end{align}
dove $f(x)$ è una funzione a valori reali nelle variabili $x$, $D$ è il dominio di x e $S$ un insieme finito di vincoli. In generale,  
$x$ è una tupla ($x_1,...,x_n$) e $D$ è un prodotto cartesiano $D_1 \times ... \times D_n$, e vale $x_j \, \in \, D_j$. 

Un problema nella forma (\ref{eq:opt}) è intrattabile, ovvero non esistono algoritmi efficienti (o non esiste proprio
alcun algoritmo) per la sua risoluzione. Si rende quindi necessario considerare dei casi particolari di questa formulazione, i quali 
possiedono determinate proprietà che possono essere sfruttate nella definizione di algoritmi ad-hoc.
\newpage

Nelle successive sezioni di questa introduzione verranno brevemente esposti alcuni concetti teorici rilevanti nel contesto degli 
esperimenti svolti. La trattazione proseguirà poi con l'esposizione dell'impostazione utilizzata nello svolgimento degli
esperimenti, dei risultati ottenuti e di un conciso commento riguardo questi ultimi. 

\noindent
Tutto il codice sviluppato in relazione a questo elaborato è stato scritto in Python \cite{python}, linguaggio di programmazione \textit{general-purpose} di alto livello,  ed è liberamente consultabile online \cite{repository}.