%*****************
%* nuovi comandi *
%*****************

\newcommand{\abs}[1]{\left|#1\right|}                               % modulo
\newcommand{\dato}{\left|\right.}                                   % probabilit\`{a} condizionata
\newcommand{\fun}[1]{\mathrm{#1}}                                   % stile funzione
\newcommand{\imp}{\;\;\Longrightarrow\;\;}                          % implicazione
\newcommand{\norma}[1]{\left\| #1 \right\|}                         % norma
\newcommand{\prob}[1]{\mathrm{P}\!\left[#1\right]}                  % probabilit\`{a}
\newcommand{\expect}[1]{\mathrm{E}\!\left[#1\right]}                % aspettazione
\newcommand{\sse}{\;\;\Longleftrightarrow\;\;}                      % se e solo se
\newcommand{\vect}[1]{{\boldsymbol{\mathrm{#1}}}}                   % stile vettore
\newcommand{\real}[1]{\fun{Re}\left[#1\right]}                      % parte reale
\newcommand{\imag}[1]{\fun{Im}\left[#1\right]}                      % parte immaginaria
\newcommand{\Dim}[1]{\fun{dim}\left[#1\right]}                      % dimensione di una matrice
\newcommand{\Det}[1]{\fun{det}\left[#1\right]}                      % determinante di una matrice
\newcommand{\Ker}[1]{\fun{ker}\left[#1\right]}                      % ker di una matrice
\newcommand{\rango}[1]{\fun{rango}\left[#1\right]}                  % rango di una matrice
\newcommand{\scalare}[2]{\left\langle #1, #2 \right\rangle}         % prodotto scalare
\newcommand{\blbrace}{\left  \lbrace}                               % parentesi graffa sinistra grande
\newcommand{\brbrace}{\right \rbrace}                               % parentesi graffa destra grande
\newcommand{\sinc}{\fun{sinc}}                                      % sinc
\newcommand{\rect}{\fun{rect}}                                      % rect
\newcommand{\rcos}{\fun{rcos}}                                      % rcos
\newcommand{\sgn}{\fun{sgn}}                                        % sgn
\newcommand{\N}{\mathbb{N}}                                         % insieme dei numeri naturali
\newcommand{\Z}{\mathbb{Z}}                                         % insieme dei numeri interi
\newcommand{\Q}{\mathbb{Q}}                                         % insieme dei numeri razionali
\newcommand{\R}{\mathbb{R}}                                         % insieme dei numeri reali
\newcommand{\C}{\mathbb{C}}                                         % insieme dei numeri complessi
\newcommand{\seq}[2][n]{#2_{0}, #2_{1}, \ldots, \, #2_{#1}}         % sequenza
\newcommand{\Span}[2][n]
{\fun{span} \blbrace #2_{1}, #2_{2}, \ldots, \, #2_{#1} \brbrace}   % spazio generato
\newcommand{\ddt}{\frac{\fun{d}}{\fun{dt}}}                         % derivata
\newcommand{\Div}[2]{#1 \; \mid \; #2}                              % divide
\newcommand{\MCD}[2]{\fun{MCD}\(#1, #2\)}                           % massimo comun divisore
\newcommand{\mcm}[2]{\fun{mcm}\(#1, #2\)}                           % minimo comune multiplo
\newcommand{\goodgap}{
                      \hspace{\subfigtopskip}
                      \hspace{\subfigbottomskip}
                     }                                              % interlinea opportuna per le sottofigure
\newcommand{\eng}[1]{\emph{#1}}                                   % inglese
\newcommand{\virg}[1]{``#1"}                                        % fa una citazione tra virgolette
%\newcommand{\unit}[2]($\frac{\text{#1}}{\text{#2}}$)                % unit\`{a} di misura
\newcommand{\textttvar}[1]{{\ttvar #1}}

%****************************
%* ridefinizioni di comandi *
%****************************

\renewcommand{\(}{\left(}                                     % parentesi tonda sinistra grande
\renewcommand{\)}{\right)}                                    % parentesi tonda destra grande
\renewcommand{\[}{\left[}                                     % parentesi quadra sinistra grande
\renewcommand{\]}{\right]}                                    % parentesi quadra destra grande
\renewcommand{\exp}[1]{\fun{e}^{#1}}                          % esponenziale
\renewcommand{\gcd}[2]{\fun{gcd}\(#1, #2\)}                   % massimo comun divisore
