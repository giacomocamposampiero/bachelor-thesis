\section{Prova di sottosezione}
La struttura utilizzata in questo template non è obbligatoria, però ritengo che sia molto comoda per evitare di scrivere file troppo lunghi e di avere un controllo migliore sulla struttura. Questa prevede di scrivere l'introduzione al capitolo in un file salvato nella cartella principale e di sviluppare le sezioni all'interno di una cartella. Esempio di richiamo ad un riferimento \cite{BibConverter-demo}.\\

\noindent
La struttura utilizzata in questo template non è obbligatoria, però ritengo che sia molto comoda per evitare di scrivere file troppo lunghi e di avere un controllo migliore sulla struttura. Questa prevede di scrivere l'introduzione al capitolo in un file salvato nella cartella principale e di sviluppare le sezioni all'interno di una cartella. Esempio di richiamo ad un riferimento \cite{BibConverter-demo}.\\

\noindent
La struttura utilizzata in questo template non è obbligatoria, però ritengo che sia molto comoda per evitare di scrivere file troppo lunghi e di avere un controllo migliore sulla struttura. Questa prevede di scrivere l'introduzione al capitolo in un file salvato nella cartella principale e di sviluppare le sezioni all'interno di una cartella. Esempio di richiamo ad un riferimento \cite{BibConverter-demo}.\\

\noindent
La struttura utilizzata in questo template non è obbligatoria, però ritengo che sia molto comoda per evitare di scrivere file troppo lunghi e di avere un controllo migliore sulla struttura. Questa prevede di scrivere l'introduzione al capitolo in un file salvato nella cartella principale e di sviluppare le sezioni all'interno di una cartella. Esempio di richiamo ad un riferimento \cite{BibConverter-demo}.\\

\noindent
La struttura utilizzata in questo template non è obbligatoria, però ritengo che sia molto comoda per evitare di scrivere file troppo lunghi e di avere un controllo migliore sulla struttura. Questa prevede di scrivere l'introduzione al capitolo in un file salvato nella cartella principale e di sviluppare le sezioni all'interno di una cartella. Esempio di richiamo ad un riferimento \cite{BibConverter-demo}.\\

\noindent
La struttura utilizzata in questo template non è obbligatoria, però ritengo che sia molto comoda per evitare di scrivere file troppo lunghi e di avere un controllo migliore sulla struttura. Questa prevede di scrivere l'introduzione al capitolo in un file salvato nella cartella principale e di sviluppare le sezioni all'interno di una cartella. Esempio di richiamo ad un riferimento \cite{BibConverter-demo}.\\

\noindent
La struttura utilizzata in questo template non è obbligatoria, però ritengo che sia molto comoda per evitare di scrivere file troppo lunghi e di avere un controllo migliore sulla struttura. Questa prevede di scrivere l'introduzione al capitolo in un file salvato nella cartella principale e di sviluppare le sezioni all'interno di una cartella. Esempio di richiamo ad un riferimento \cite{BibConverter-demo}.\\

\subsection{Prova sottosezione 21}
\noindent
La struttura utilizzata in questo template non è obbligatoria, però ritengo che sia molto comoda per evitare di scrivere file troppo lunghi e di avere un controllo migliore sulla struttura. Questa prevede di scrivere l'introduzione al capitolo in un file salvato nella cartella principale e di sviluppare le sezioni all'interno di una cartella. Esempio di richiamo ad un riferimento \cite{BibConverter-demo}.\\

\noindent
La struttura utilizzata in questo template non è obbligatoria, però ritengo che sia molto comoda per evitare di scrivere file troppo lunghi e di avere un controllo migliore sulla struttura. Questa prevede di scrivere l'introduzione al capitolo in un file salvato nella cartella principale e di sviluppare le sezioni all'interno di una cartella. Esempio di richiamo ad un riferimento \cite{BibConverter-demo}.\\

\noindent
La struttura utilizzata in questo template non è obbligatoria, però ritengo che sia molto comoda per evitare di scrivere file troppo lunghi e di avere un controllo migliore sulla struttura. Questa prevede di scrivere l'introduzione al capitolo in un file salvato nella cartella principale e di sviluppare le sezioni all'interno di una cartella. Esempio di richiamo ad un riferimento \cite{BibConverter-demo}.\\

\noindent
La struttura utilizzata in questo template non è obbligatoria, però ritengo che sia molto comoda per evitare di scrivere file troppo lunghi e di avere un controllo migliore sulla struttura. Questa prevede di scrivere l'introduzione al capitolo in un file salvato nella cartella principale e di sviluppare le sezioni all'interno di una cartella. Esempio di richiamo ad un riferimento \cite{BibConverter-demo}.\\

\noindent
La struttura utilizzata in questo template non è obbligatoria, però ritengo che sia molto comoda per evitare di scrivere file troppo lunghi e di avere un controllo migliore sulla struttura. Questa prevede di scrivere l'introduzione al capitolo in un file salvato nella cartella principale e di sviluppare le sezioni all'interno di una cartella. Esempio di richiamo ad un riferimento \cite{BibConverter-demo}.\\

\noindent
La struttura utilizzata in questo template non è obbligatoria, però ritengo che sia molto comoda per evitare di scrivere file troppo lunghi e di avere un controllo migliore sulla struttura. Questa prevede di scrivere l'introduzione al capitolo in un file salvato nella cartella principale e di sviluppare le sezioni all'interno di una cartella. Esempio di richiamo ad un riferimento \cite{BibConverter-demo}.\\

\noindent
La struttura utilizzata in questo template non è obbligatoria, però ritengo che sia molto comoda per evitare di scrivere file troppo lunghi e di avere un controllo migliore sulla struttura. Questa prevede di scrivere l'introduzione al capitolo in un file salvato nella cartella principale e di sviluppare le sezioni all'interno di una cartella. Esempio di richiamo ad un riferimento \cite{BibConverter-demo}.\\

\noindent
La struttura utilizzata in questo template non è obbligatoria, però ritengo che sia molto comoda per evitare di scrivere file troppo lunghi e di avere un controllo migliore sulla struttura. Questa prevede di scrivere l'introduzione al capitolo in un file salvato nella cartella principale e di sviluppare le sezioni all'interno di una cartella. Esempio di richiamo ad un riferimento \cite{BibConverter-demo}.\\

\noindent
La struttura utilizzata in questo template non è obbligatoria, però ritengo che sia molto comoda per evitare di scrivere file troppo lunghi e di avere un controllo migliore sulla struttura. Questa prevede di scrivere l'introduzione al capitolo in un file salvato nella cartella principale e di sviluppare le sezioni all'interno di una cartella. Esempio di richiamo ad un riferimento \cite{BibConverter-demo}.\\

\noindent
La struttura utilizzata in questo template non è obbligatoria, però ritengo che sia molto comoda per evitare di scrivere file troppo lunghi e di avere un controllo migliore sulla struttura. Questa prevede di scrivere l'introduzione al capitolo in un file salvato nella cartella principale e di sviluppare le sezioni all'interno di una cartella. Esempio di richiamo ad un riferimento \cite{BibConverter-demo}.\\

\noindent
La struttura utilizzata in questo template non è obbligatoria, però ritengo che sia molto comoda per evitare di scrivere file troppo lunghi e di avere un controllo migliore sulla struttura. Questa prevede di scrivere l'introduzione al capitolo in un file salvato nella cartella principale e di sviluppare le sezioni all'interno di una cartella. Esempio di richiamo ad un riferimento \cite{BibConverter-demo}.\\

\noindent
La struttura utilizzata in questo template non è obbligatoria, però ritengo che sia molto comoda per evitare di scrivere file troppo lunghi e di avere un controllo migliore sulla struttura. Questa prevede di scrivere l'introduzione al capitolo in un file salvato nella cartella principale e di sviluppare le sezioni all'interno di una cartella. Esempio di richiamo ad un riferimento \cite{BibConverter-demo}.\\

\noindent
La struttura utilizzata in questo template non è obbligatoria, però ritengo che sia molto comoda per evitare di scrivere file troppo lunghi e di avere un controllo migliore sulla struttura. Questa prevede di scrivere l'introduzione al capitolo in un file salvato nella cartella principale e di sviluppare le sezioni all'interno di una cartella. Esempio di richiamo ad un riferimento \cite{BibConverter-demo}.\\

\noindent
La struttura utilizzata in questo template non è obbligatoria, però ritengo che sia molto comoda per evitare di scrivere file troppo lunghi e di avere un controllo migliore sulla struttura. Questa prevede di scrivere l'introduzione al capitolo in un file salvato nella cartella principale e di sviluppare le sezioni all'interno di una cartella. Esempio di richiamo ad un riferimento \cite{BibConverter-demo}.\\

\noindent
La struttura utilizzata in questo template non è obbligatoria, però ritengo che sia molto comoda per evitare di scrivere file troppo lunghi e di avere un controllo migliore sulla struttura. Questa prevede di scrivere l'introduzione al capitolo in un file salvato nella cartella principale e di sviluppare le sezioni all'interno di una cartella. Esempio di richiamo ad un riferimento \cite{BibConverter-demo}.\\

\noindent
La struttura utilizzata in questo template non è obbligatoria, però ritengo che sia molto comoda per evitare di scrivere file troppo lunghi e di avere un controllo migliore sulla struttura. Questa prevede di scrivere l'introduzione al capitolo in un file salvato nella cartella principale e di sviluppare le sezioni all'interno di una cartella. Esempio di richiamo ad un riferimento \cite{BibConverter-demo}.\\

\noindent
La struttura utilizzata in questo template non è obbligatoria, però ritengo che sia molto comoda per evitare di scrivere file troppo lunghi e di avere un controllo migliore sulla struttura. Questa prevede di scrivere l'introduzione al capitolo in un file salvato nella cartella principale e di sviluppare le sezioni all'interno di una cartella. Esempio di richiamo ad un riferimento \cite{BibConverter-demo}.\\

\noindent
La struttura utilizzata in questo template non è obbligatoria, però ritengo che sia molto comoda per evitare di scrivere file troppo lunghi e di avere un controllo migliore sulla struttura. Questa prevede di scrivere l'introduzione al capitolo in un file salvato nella cartella principale e di sviluppare le sezioni all'interno di una cartella. Esempio di richiamo ad un riferimento \cite{BibConverter-demo}.\\

\noindent
La struttura utilizzata in questo template non è obbligatoria, però ritengo che sia molto comoda per evitare di scrivere file troppo lunghi e di avere un controllo migliore sulla struttura. Questa prevede di scrivere l'introduzione al capitolo in un file salvato nella cartella principale e di sviluppare le sezioni all'interno di una cartella. Esempio di richiamo ad un riferimento \cite{BibConverter-demo}.\\