In questo lavoro è stato affrontato lo studio di un particolare problema di ottimizzazione, il problema del vertex cover minimo. In particolare, ne è stato sperimentalmente analizzato l'andamento della complessità di risoluzione al variare dei parametri utilizzati nella generazione, per ognuno dei modelli di generazione di grafi casuali considerati.

Per alcuni modelli è stato possibile individuare degli andamenti della complessità nella risoluzione del problema chiari e coerenti, come ad esempio nel caso dei grafi di Barabási-Albert e di Erdős–Rényi. Per altre tipologie di modelli di generazione di grafi casuali l'individuazione di un andamento chiaro è stato possibile solo in parte (grafi di Watts-Strogatz), mentre per altri ancora non è stata possibile in alcuna misura (grafi regolari). 

Gli esperimenti presentati in questo lavoro sono stati tuttavia svolti su di un insieme di combinazioni di parametri relativamente ristretto. Un interessante sviluppo delle sperimentazioni potrebbe consistere nell'estensione dell'insieme di combinazioni di parametri utilizzati, per verificare la validità delle considerazioni proposte in questo lavoro ed eventualmente per produrne di nuove, e l'estensione del time limit imposto al risolutore CPLEX per la soluzione dei problemi, al fine di aumentare la discriminazione nell'insieme di istanze che al momento vanno in time limit. 