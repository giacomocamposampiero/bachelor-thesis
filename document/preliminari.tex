In questo capitolo sono stati raccolti alcuni concetti teorici di ottimizzazione legati al lavoro svolto. È stata come prima cosa data una definizione formale al concetto di problema di ottimizzazione. Si è proseguito poi con un approfondimento sulla programmazione lineare intera e su due algoritmi utilizzati nella risoluzione di problemi di ottimizzazione, \textit{branch-and-bound} e \textit{branch-and-cut}. Infine, è stata esposta una definizione formale al problema di ottimizzazione studiato in questo lavoro, il problema di copertura dei vertici minimo. Molte delle definizioni in questo capitolo sono ispirate alle dispense del corso di \textsl{Modelli e Software per l'Ottimizzazione Discreta}, tenuto dal professore Domenico Salvagnin \cite{salvagnin1}\cite{salvagnin2}.

\section{Problema di ottimizzazione}
Un problema di ottimizzazione può essere formulato come 
\begin{align}
	\label{eq:opt}
	\begin{array}{l}
      \text{min(\textit{or} max)}f(x)\\
      S	\\
      x \in D
    \end{array}
\end{align}
dove $f(x)$ è una funzione a valori reali nelle variabili $x$, $D$ è il dominio di $x$ e $S$ un insieme finito di vincoli. In generale,  
$x$ è una tupla ($x_1,...,x_n$) e $D$ è un prodotto cartesiano $D_1 \times ... \times D_n$, e vale $x_j \, \in \, D_j$. 

Un problema nella forma (\ref{eq:opt}) è intrattabile, ovvero non esistono algoritmi efficienti (o non esiste proprio
alcun algoritmo) per la sua risoluzione. Si rende quindi necessario considerare dei casi particolari di questa formulazione, i quali 
possiedono determinate proprietà che possono essere sfruttate nella definizione di algoritmi risolutivi ad-hoc.

\newpage
\section{Programmazione lineare intera}
Uno dei casi particolari della formulazione generale (\ref{eq:opt}) a cui si è precedentemente fatto riferimento viene trattato dalla programmazione lineare intera.
Un problema di programmazione lineare intera consiste nella minimizzazione (o massimizzazione) di una funzione lineare soggetta ad un 
numero finito di vincoli lineari, con in aggiunta il vincolo che alcune delle variabili del problema debbano assumere valori interi. 
In generale, il problema può quindi essere riformulato come
\begin{align*}
	\begin{array}{l}
      \text{min} \, cx\\
      a_i x \sim b_i \qquad i=1,...,m \\
      l_j \leq x_j \leq u_j \qquad j=1,...,n =N \\
      x_j \in \mathbb{Z}  \qquad \forall j \in J \subseteq N = {1,...,n}	
	\end{array}
\end{align*}
\indent
Se $J=N$ si parla di programmazione lineare intera pura, altrimenti di programmazione lineare intera mista (o MIP, dall'inglese 
\textit{Mixed Integer Programming}).

La programmazione lineare intera restringe quindi notevolmente il tipo di vincoli a disposizione nel
processo di formalizzazione matematica del problema, determinando una maggior difficoltà in fase di modellazione.
Tuttavia, questo non comporta eccessive restrizioni, almeno per la MIP, sulle tipologie di problemi che possono essere formulati 
secondo questo paradigma. Alcuni esempi classici di problemi risolvibili mediante MIP sono \textit{knapsack}, problemi di 
\textit{scheduling}, \textit{facility location} e, naturalmente, \textit{minimum vertex cover}.
Inoltre, l'introduzione dei vincoli di linearità ed interezza comporta notevoli vantaggi nella definizione ed
implementazione di algoritmi risolutivi, alcuni dei quali verranno descritti più nel dettaglio nelle seguenti sezioni.

\subsection{Algoritmo Branch and Bound}
L'algoritmo branch-and-bound (B\&B) è un algoritmo di ottimizzazione generica basato sull'enumerazione dell'insieme delle soluzioni
ammissibili di un problema di ottimizzazione combinatoria, introdotto nel 1960 da A. H. Land e A. G. Doig \cite{10.2307/1910129}. 

Questo algoritmo permette di gestire il problema dell'esplosione combinatoria mediante il \textit{pruning} di intere porzioni dello spazio delle soluzioni, che può essere effettuato quando si riesce a dimostrare che queste ultime non contengono soluzioni migliori di 
quelle note. Branch-and-bound implementa inoltre una strategia \textit{divide and conquer}, che permette di partizionare lo spazio di 
ricerca e di risolvere ognuna delle partizioni ottenute separatamente. Viene riportata di seguito una breve descrizione dell'algoritmo.

Sia $F$ l'insieme delle soluzioni ammissibili di un problema di minimizzazione (simmetrico nel caso della massimizzazione, a meno di un 
cambio di segno della funzione obiettivo), $c \, : \, F \, \rightarrow \mathbb{R} $ la funzione obiettivo e $\Bar{x} \in F$ una 
soluzione ammissibile nota, generata mediante euristiche o mediante assegnazioni casuali. Sia $z=f(\Bar{x})$ il costo di tale soluzione
nota, anche detto \textit{incumbent}, che rappresenta per sua natura un limite superiore al valore della soluzione ottima. 

L'algoritmo branch-and-bound esegue un'iniziale fase di \textit{bounding} in cui uno o più vincoli del problema vengono rilassati, allargando di conseguenza l'insieme delle possibili soluzioni $G \supseteq F$. La soluzione di questo rilassamento, se esiste,
rappresenta un \textit{lower bound} alla soluzione ottima del problema iniziale. Se la soluzione di tale rilassamento appartiene a $F$
o ha costo uguale all'attuale \textit{incumbent}, allora l'algoritmo termina, in quanto si è trovata una soluzione ottima del problema.
Se il rilassamento risulta essere impossibile, possiamo anche in questo caso terminare la ricerca di una soluzione e 
concludere che anche il problema di partenza è impossibile.

Nel caso in cui invece una soluzione al rilassamento esiste ma non è contenuta nell'insieme delle soluzioni ammissibili $F$,
l'algoritmo procede con l'identificare una separazione $F^*$ di $F$, ossia un insieme finito di sottoinsiemi tali che
\begin{align*}
\bigcup_{F_i \in F^*} F_{i} = F
\end{align*}
Questa fase, detta di \textit{branching}, è giustificata dal fatto che la soluzione ottima dell'intero problema è data dalla minima
tra le soluzioni delle varie separazioni $F_i \in F^*$. $F^*$ è spesso, anche se non necessariamente, una partizione dell'insieme 
iniziale $F$. A questo punto, tutti i figli di $F$ vengono aggiunti alla coda dei sotto-problemi da processare.

L'algoritmo procede quindi con il selezionare un sotto-problema $P_i$ dalla coda un rilassamento. Quattro sono i possibili casi:
\begin{itemize}
\itemsep-0.5em 
\item Se si trova una soluzione $\in F$ migliore dell'attuale incumbent, quest'ultimo viene 
sostituito dalla soluzione trovata e si procede con lo studio di un altro sotto-problema.
\item Se il rilassamento del sotto-problema non ammette soluzione, allora si smette di esplorare l'intero sotto-albero a lui associato 
nello spazio di ricerca (\textit{pruning by infeasibility}).
\item Altrimenti, si confronta la soluzione trovata con il valore corrente dell'\textit{upper-bound} dato dall'incumbent; se
quest'ultimo è minore
della soluzione del rilassamento trovata, è possibile anche in questo caso smettere di esplorare il sotto-albero associato al
sotto-problema corrente,
in quanto non può portare ad una soluzione migliore di quella che già si conosce (\textit{pruning by optimality}).
\item Infine, se non è stato possibile scartare o concludere la visita del sotto-albero associato a $P_i$ in alcun modo, è necessario
eseguire nuovamente il \textit{branching}, aggiungendo i nuovi sotto-problemi alla lista dei sotto-problemi da processare.
\end{itemize}
L'algoritmo prosegue nella selezione di sotto-problemi finché la lista di questi ultimi non si svuota. Quando ciò avviene, la soluzione
rappresentata dall'attuale \textit{incumbent} è la soluzione ottima al problema iniziale.

Quella appena descritta rappresenta una formulazione generica dell'algoritmo B\&B. Questa formulazione può essere
tuttavia specializzata nella risoluzione di problemi MIP con relativa semplicità, agendo sulle condizioni che regolano 
\textit{bounding} e \textit{branching}. Per quanto riguarda il primo, la scelta più diffusa consiste nel rilassamento del vincolo di interezza.
Se la soluzione del rilassamento non è intera, una possibile separazione in sotto-problemi può essere fatta considerando la partizione
\begin{align*}
x_j \leq \floor*{x_j^*} \vee x_j \geq \ceil*{x_j^*}
\end{align*}
\indent
Infine, è importante notare come, per costruzione, ogni soluzione trovata dall'algoritmo è migliore dell'incumbent e, di conseguenza, l'andamento dell'
upper bound del problema è strettamente decrescente. I lower bound dei singoli sotto-problemi non hanno invece, al contrario degli 
upper bound, valenza globale. Derivare un lower bound globale è comunque possibile, considerando il minimo tra tutti
i lower bound dei sotto-problemi ancora aperti. Avendo a disposizione in ogni momento entrambi i bound del problema, 
è possibile quindi valutare la bontà della soluzione provvisoria in qualsiasi momento. 

\subsection{Algoritmo Branch and Cut}
L'algoritmo \textit{branch-and-cut} (B\&C) rappresenta una versione migliorata dell'algoritmo branch-and-bound, introdotta nel 1987
da M. Padberg e G. Rinaldi \cite{PADBERG19871} e ideata appositamente per la risoluzione di problemi MIP. 

L'algoritmo branch-and-cut è un ibrido tra branch-and-bound, trattato in precedenza, e un algoritmo
a piani di taglio puro, in cui la soluzione è ottenuta mediante raffinazioni successive dello spazio delle soluzioni mediante la
progressiva aggiunti di vincoli.
Queste due tecniche si rafforzano a vicenda, contribuendo al raggiungimento di prestazioni complessive superiori a quelle
che otterrebbe ciascuna di esse singolarmente.

L'idea alla base di questo algoritmo è quella di "rafforzare", per ogni sotto-problema, la formulazione associata al suo rilassamento
lineare mediante la generazione di piani di taglio. I vantaggi a livello risolutivo sono diversi, tra cui una maggior probabilità di
ottenere soluzioni intere al rilassamento lineare o, in alternativa, di ottenere lower bound migliori e quindi più discriminanti in 
fase di pruning. 

Nonostante l'idea alla base di questo approccio sia relativamente semplice, l'implementazione dell'algoritmo B\&C è tutt'altro che
scontata e richiede l'esistenza di procedure efficienti per la risoluzione del seguente problema di \textsl{separazione}: data una
soluzione frazionaria {$x^*$}, trovare una diseguaglianza valida $\alpha^Tx\leq \alpha_0$ (se esiste) violata da $x^*$, cioè
tale che $\alpha^T$.

\section{Vertex cover}
Il problema di \textit{vertex cover} (anche detto di \textsl{copertura dei vertici}) è un problema di ottimizzazione combinatoria 
che consiste nel trovare il minimo vertex cover di un grafo, ossia il più piccolo insieme di nodi del grafo tale che almeno uno dei due 
vertici di ogni arco sia contenuto in questo insieme. Trovare il vertex cover minimo di un generico grafo è un problema 
\textsl{NP-completo}, ovvero non esistono algoritmi in grado di trovarne una soluzione in un tempo polinomiale. In questo elaborato è
stata considerata la formulazione indiretta e \textit{unweighted} del problema, in cui gli archi non sono direzionati e in cui i nodi del grafo hanno tutti peso uguale e pari ad $1$.

Formalmente, il vertex cover $V'$ di un grafo $G=(V,E)$ può essere definito come un sotto-insieme dei vertici del grafo $V$ tale che
\begin{align*}
uv \in E \Rightarrow u \in V' \vee v \in V'
\end{align*}
Definita $\tau=\mid V' \mid$ la cardinalità del vertex cover, il vertex cover minimo può essere definito formalmente come
\vspace{-0.5cm}
\begin{align*}
V_{min}' \, = \, \arg\min_{\tau} \, V' 
\end{align*} 

